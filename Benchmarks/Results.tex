\documentclass[12pt, landscape]{article}
\usepackage[margin=0.75in]{geometry}
\usepackage{mhchem}
\usepackage{url}

\begin{document}
This document details the input parameters needed to reproduce all results in our forthcoming publication (\url{https://arxiv.org/abs/2005.00654}). More methodological details for each of these methods are provided in the paper. The statistics reported for each calculation can be obtained using the \texttt{calc\_stats.py} Python script provided in this directory.

\section*{FCI-FRI calculations}

The following results were obtained using the systematic FCI-FRI method without the semi-stochastic extension, using the original heat-bath Power-Pitzer factorization. 

\subsection*{Ne (aug-cc-pVDZ)}
These results are for the Neon atom in the aug-cc-pVDZ basis. The Hartree-Fock data for this system, generated using pyscf, can be found in the directory \texttt{Input\_Data/Neon\_augccpvdz}. As an example, the results in the second line of this table were generated by running the following command:

\texttt{mpirun -n 4 ./build/FRIES\_bin/frisys\_mol --hf\_path Input\_Data/Neon\_augccpvdz/ --target 100000 --distribution HB --vec\_nonz 100000 --mat\_nonz 100000 --max\_dets 50000 --initiator 0.5 --ini\_vec Input\_Data/Neon\_augccpvdz/cisd\_}
\\~\\
\begin{tabular}{c|c|c|c|c|c|c|c|c|c|c}
Vec. nonz. & Matr. nonz. & Target$^1$ & $n_a^2$ & Trial vec.$^3$ & Ini vec. & Burn-in & Iterations & Mean $\pm 2 \sigma$ (m$E_h$)$^4$ & Efficiency ($E_h^{-2}$) & Figures$^5$ \\ \hline
100k & 100k & 100k & 0.0 & HF & CISD & 40k & 1M & $4.4561 \pm 16.5977$ & 0.0151 & 1 \\
100k & 100k & 100k & 0.5 & HF & CISD  & 40k & 1M & $2.3959 \pm 4.4132$ & 0.2139 & 1 \\
100k & 100k & 100k & 1.0 & HF & CISD  & 40k & 1M & $0.0223 \pm 0.0070$ & 85802 & 1, 2, 4 \\
100k & 100k & 100k & 1.5 & HF & CISD  & 40k & 1M & $0.0282 \pm 0.0064$ & 100190 & 1 \\
100k & 100k & 100k & 2.0 & HF & CISD  & 40k & 1M & $0.0321 \pm 0.0065$ & 97480 & 1 \\
50k & 50k & 50k & 1 & HF & CISD &  40k & 1M & $0.0103 \pm 0.0095$ & 46368 & 2, 4 \\
150k & 150k & 150k & 1 & HF & CISD &  40k & 1M & $0.0102 \pm 0.0055$ & 135837 & 2, 3, 4 \\
200k & 200k & 200k & 1 & HF & CISD &  40k & 1M & $0.0052 \pm 0.0095$ & 206821 & 2, 3, 4 \\
250k & 250k & 250k & 1 & HF & CISD &  40k & 1M & $0.0038 \pm 0.0040$ & 259087 & 2, 3 \\
24k & 52k & 50k & 3 & HF & CISD & 40k & 1M & $0.0038 \pm 0.0101$ & 40798 & 5 \\
42k & 104k & 100k &  3 & HF & CISD & 40k & 1M & $ 0.0196 \pm 0.0064$ & 100854 & 5 \\
58k & 157k & 150k & 3 & HF & CISD & 40k & 1M & $0.0247 \pm 0.0051$ & 158244 & 5 \\
73k & 209k & 200k & 3 & HF & CISD & 40k & 1M & $0.0251 \pm 0.0043$ & 229864 & 5 \\
\end{tabular}
$^1$ Indicates the target one-norm of the iterates. \\
$^2$ $n_a$ indicates the initiator threshold. \\
$^3$ Because no trial vector was specified in the input options, Hartree-Fock was used.\\
$^4$ The exact FCI ground-state energy was subtracted from these estimates. \\
$^5$ Indicates the figures in our paper in which the data appears.

\subsection*{\ce{H2O} (cc-pVDZ)}
These results are for \ce{H2O} in the cc-pVDZ basis. The Hartree-Fock data for this system, generated using pyscf, can be found in the directory \texttt{Input\_Data/H2O\_ccpvdz}. As an example, the results in the second line of this table were generated by running the following command:

\texttt{mpirun -n 16 ./build/FRIES\_bin/frisys\_mol --hf\_path Input\_Data/H2O\_ccpvdz/ --target 1000000 --distribution HB}\\ \texttt{--vec\_nonz 1000000 --mat\_nonz 1000000 --max\_dets 100000 --initiator 0.5 --ini\_vec Input\_Data/H2O\_ccpvdz/cisd\_}
\\~\\
\begin{tabular}{c|c|c|c|c|c|c|c|c|c|c}
Vec. nonz. & Matr. nonz. & Target & $n_a$ & Trial vec. & Ini vec. & Burn-in & Iterations & Mean $\pm 2 \sigma$ (m$E_h$) & Efficiency ($E_h^{-2}$) & Figures \\ \hline
1M & 1M & 1M & 0.0 & HF & CISD & 40k & 1M & $173.9337 \pm 49.0249$ & 0.00173 & 1 \\
1M & 1M & 1M & 0.5 & HF & CISD & 40k & 1M & $2.4213 \pm 3.1589$ & 0.4176 & 1 \\
1M & 1M & 1M & 1.0 & HF & CISD & 40k & 1M & $0.0171 \pm 0.0067$ & 92963 & 1,2 \\
1M & 1M & 1M & 1.5 & HF & CISD & 40k & 1M & $0.0273 \pm 0.0064$ & 102492 & 1 \\
1M & 1M & 1M & 2.0 & HF & CISD & 40k & 1M & $0.0298 \pm 0.0069$ & 86657 & 1 \\
3M & 3M & 3M & 1 & HF & CISD & 40k & 524k & $0.0172 \pm 0.0048$ & 358168 & 2, 3, 4\\
5M & 5M & 5M & 1 & HF & CISD & 40k & 319k & $0.0124 \pm 0.0044$ & 732651 & 2, 3, 4\\
528k & 1.1M & 1M & 3 & HF & CISD & 40k & 1M & $0.0444 \pm 0.0059$ & 119583 & 5 \\
3.4M & 11.0M & 10M & 3 & HF & CISD & 40k &424k & $0.0174 \pm 0.0019$ & 2773910 & 5 \\
9.9M & 52.3M & 50M & 3 & HF & CISD & 40k & 621k & $0.0155 \pm 0.0004$ & 42184887 & 5 \\
\end{tabular}


\subsection*{\ce{N2} (equilibrium, cc-pVDZ)}
These results are for \ce{N2} at the equilibrium geometry ($r_\text{NN} = 2.068 a_0$) in the cc-pVDZ basis. The Hartree-Fock data for this system, generated using pyscf, can be found in the directory \texttt{Input\_Data/N2\_ccpvdz}. As an example, the results in the second line of this table were generated by running the following command:

\texttt{mpirun -n 20 ./build/FRIES\_bin/frisys\_mol --hf\_path Input\_Data/N2\_ccpvdz/ --target 1000000 --distribution HB} \texttt{--vec\_nonz 1000000 --mat\_nonz 1000000 --max\_dets 100000 --initiator 0.5 --ini\_vec Input\_Data/N2\_ccpvdz/cisd\_}
\\~\\
\begin{tabular}{c|c|c|c|c|c|c|c|c|c|c}
Vec. nonz. & Matr. nonz. & Target & $n_a$ & Trial vec. & Ini vec. & Burn-in & Iterations & Mean $\pm 2 \sigma$ (m$E_h$) & Efficiency ($E_h^{-2}$) & Figures \\ \hline
1M & 1M & 1M & 0.0 & HF & CISD & 40k & 1M & $1593.6500 \pm 8414.6232$ & 5.88e-8 & 1 \\
1M & 1M & 1M & 0.5 & HF & CISD & 40k & 1M & $284.2253 \pm 83.8154$ & 0.000593 & 1 \\
1M & 1M & 1M & 1.0 & HF & CISD & 40k & 1M & $0.0275 \pm 0.0208$ & 9592 & 1,2 \\
1M & 1M & 1M & 1.5 & HF & CISD & 40k & 1M & $0.0645 \pm 0.0161$ & 16103 & 1 \\
1M & 1M & 1M & 2.0 & HF & CISD & 40k & 1M & $0.1103 \pm 0.0154$ & 17610 & 1 \\
3M & 3M & 3M & 1 & HF & CISD & 40k & 1M & $0.0080 \pm 0.0101$ & 40632 & 2, 3, 4\\
5M & 5M & 5M & 1 & HF & CISD & 40k & 760k & $0.0087 \pm 0.0082$ & 83285 & 2, 3, 4\\
652k & 1.2M & 1M &  3 & HF & CISD & 40k & 1M & $0.1431 \pm 0.0127$ & 25676 & 5 \\
4.8M & 11.0M & 10M & 3 & HF & CISD & 40k & 579k & $0.0167 \pm 0.0046$ & 348800 & 5 \\
18.2M & 53.4M & 50M & 3 & HF & CISD & 40k & 272k & $0.0074 \pm 0.0022$ & 3684000 & 5\\
\end{tabular}

\subsection*{Ne (cc-pVQZ)}
These results are for Ne in the cc-pVQZ basis. The Hartree-Fock data for this system, generated using pyscf, can be found in the ZIP archive at \texttt{Input\_Data/Neon\_ccpvqz.zip}. After it is unzipped, the results in this table can be generated by running the following command:

\texttt{mpirun -n 10 ./build/FRIES\_bin/frisys\_mol --hf\_path Input\_Data/Neon\_ccpvqz/ --target 500000 --distribution HB} \texttt{--vec\_nonz 500000 --mat\_nonz 500000 --max\_dets 60000 --initiator 1}
\\~\\
\begin{tabular}{c|c|c|c|c|c|c|c|c|c|c}
Vec. nonz. & Matr. nonz. & Target & $n_a$ & Trial vec. & Ini vec.$^1$ & Burn-in & Iterations & Mean $\pm 2 \sigma$ (m$E_h$) & Efficiency ($E_h^{-2}$) & Figures \\ \hline
500k & 500k & 500k & 1 & HF & HF & 40k & 1M & $-333.3856 \pm 0.0264$ & 5994 & - \\
\end{tabular}
$^1$ Because no initial vector was specified in the input options, Hartree-Fock was used.\\

\subsection*{\ce{N2} (stretched, cc-pVDZ)}
These results are for stretched \ce{N2} ($r_\text{NN} = 4.2 a_0$) in the cc-pVDZ basis. The Hartree-Fock data for this system, generated using pyscf, can be found in the directory \texttt{Input\_Data/N2\_str\_ccpvdz}. The results in this table were generated by running the following command:

\texttt{mpirun -n 20 ./build/FRIES\_bin/frisys\_mol --hf\_path Input\_Data/N2\_str\_ccpvdz/ --target 5000000 --distribution HB} \texttt{--vec\_nonz 5000000 --mat\_nonz 5000000 --max\_dets 400000 --initiator 3 --ini\_vec Input\_Data/N2\_str\_ccpvdz/cisd\_}
\\~\\
\begin{tabular}{c|c|c|c|c|c|c|c|c|c|c}
Vec. nonz. & Matr. nonz. & Target & $n_a$ & Trial vec. & Ini vec. & Burn-in & Iterations & Mean $\pm 2 \sigma$ (m$E_h$) & Efficiency ($E_h^{-2}$) & Figures \\ \hline
500k & 500k & 500k & 3 & HF & CISD & 80k & 350k & $0.0234 \pm 0.1140$ & 1140 & - \\
\end{tabular}

\section*{Semi-stochastic FCI-FRI calculations}

The following results were obtained using the systematic FCI-FRI method with the semi-stochastic extension, using the original heat-bath Power-Pitzer factorization. 

\subsection*{Ne (aug-cc-pVDZ)}
These results are for the Neon atom in the aug-cc-pVDZ basis, using the same Hartree-Fock data described above. As an example, the results in the first line of this table were generated by running the following command:

\texttt{mpirun -n 4 ./build/FRIES\_bin/frisys\_mol --hf\_path Input\_Data/Neon\_augccpvdz/ --target 150000 --distribution HB --vec\_nonz 150000 --mat\_nonz 150000 --max\_dets 50000 --initiator 1.0 --ini\_vec Input\_Data/Neon\_augccpvdz/cisd\_} \texttt{--det\_space Input\_Data/Neon\_augccpvdz/50\_big\_cisd\_dets.txt}
\\~\\
\begin{tabular}{c|c|c|c|c|c|c|c|c|c}
Nonz.$^1$ & $n_a^2$ & Trial vec. & Ini vec. & Deterministic subspace$^3$ & Burn-in & Iterations & Mean $\pm 2 \sigma$ (m$E_h$) & Efficiency ($E_h^{-2}$) & Figures \\ \hline
150k & 1 & HF & CISD & 50 largest CISD & 40k & 1M & $0.0113 \pm 0.0048$ & 179568 & 3 \\
200k & 1 & HF & CISD & 50 largest CISD & 40k & 1M & $0.0038 \pm 0.0039$ & 276718 & 3 \\
250k & 1 & HF & CISD & 50 largest CISD & 40k & 1M & $0.0041 \pm 0.0034$ & 351606 & 3 \\
150k & 1 & HF & CISD & 50 smallest CISD & 40k & 1M & $0.0146 \pm 0.0069$ & 86276 & 3 \\
200k & 1 & HF & CISD & 50 smallest CISD & 40k & 1M & $0.0042 \pm 0.0056$ & 134809 & 3 \\
250k & 1 & HF & CISD & 50 smallest CISD & 40k & 1M & $0.0015 \pm 0.0046$ & 351606 & 3 \\
\end{tabular}
\\
$^1$ Indicates the number of nonzero elements used in matrix and vector compressions. This number was also used as the target one-norm for all calculations in this section. \\
$^2$ Indicates the initiator threshold. \\
$^3$ The deterministic subspace was constructed from the Slater determinants corresponding to the largest- or smallest-magnitude elements in the CISD ground-state eigenvector.

\subsection*{\ce{H2O} (cc-pVDZ)}
These results are for \ce{H2O} in the cc-pVDZ basis, using the same Hartree-Fock data described above. As an example, the results in the first line of this table were generated by running the following command:

\texttt{mpirun -n 16 ./build/FRIES\_bin/frisys\_mol --hf\_path Input\_Data/H2O\_ccpvdz/ --target 1000000 --distribution HB}\\ \texttt{--vec\_nonz 1000000 --mat\_nonz 1000000 --max\_dets 100000 --initiator 1.0 --ini\_vec Input\_Data/H2O\_ccpvdz/cisd\_ --det\_space Input\_Data/H2O\_ccpvdz/150\_big\_cisd\_dets.txt}
\\~\\
\begin{tabular}{c|c|c|c|c|c|c|c|c|c}
Nonz. & $n_a$ & Trial vec. & Ini vec. & Deterministic subspace & Burn-in & Iterations & Mean $\pm 2 \sigma$ (m$E_h$) & Efficiency ($E_h^{-2}$) & Figures \\ \hline
1M & 1.0 & HF & CISD & 150 largest CISD & 40k & 1M & $0.0440 \pm 0.0203$ & 10107 & 3 \\
3M & 1.0 & HF & CISD & 150 largest CISD & 40k & 1M & $0.0135 \pm 0.0035$ & 344228 & 3 \\
5M & 1.0 & HF & CISD & 150 largest CISD & 40k & 661k & $0.0178 \pm 0.0029$ & 777820 & 3 \\
1M & 1.0 & HF & CISD & 150 smallest CISD & 40k & 1M & $0.0247 \pm 0.0102$ & 40239 & 3 \\
3M & 1.0 & HF & CISD & 150 smallest CISD & 40k & 590k & $0.0178 \pm 0.0054$ & 249487 & 3 \\
5M & 1.0 & HF & CISD & 150 smallest CISD & 40k & 642k & $0.0153 \pm 0.0042$ & 385820 & 3 \\
\end{tabular}


\subsection*{\ce{N2} (equilibrium, cc-pVDZ)}
These results are for equilibrium \ce{N2} in the cc-pVDZ basis, using the same Hartree-Fock data described above. As an example, the results in the first line of this table were generated by running the following command:

\texttt{mpirun -n 20 ./build/FRIES\_bin/frisys\_mol --hf\_path Input\_Data/N2\_ccpvdz/ --target 1000000 --distribution HB}\\ \texttt{--vec\_nonz 1000000 --mat\_nonz 1000000 --max\_dets 100000 --initiator 1.0 --ini\_vec Input\_Data/N2\_ccpvdz/cisd\_ --det\_space Input\_Data/N2\_ccpvdz/150\_big\_cisd\_dets.txt}
\\~\\
\begin{tabular}{c|c|c|c|c|c|c|c|c|c}
Nonz. & $n_a$ & Trial vec. & Ini vec. & Deterministic subspace & Burn-in & Iterations & Mean $\pm 2 \sigma$ (m$E_h$) & Efficiency ($E_h^{-2}$) & Figures \\ \hline
1M & 1.0 & HF & CISD & 150 largest CISD & 40k & 1M & $0.0408 \pm 0.0172$ & 14145 & 3 \\
3M & 1.0 & HF & CISD & 150 largest CISD & 40k & 1M & $0.0085 \pm 0.0072$ & 80267 & 3 \\
5M & 1.0 & HF & CISD & 150 largest CISD & 40k & 756k & $0.0159 \pm 0.0063$ & 141279 & 3 \\
1M & 1.0 & HF & CISD & 150 smallest CISD & 40k & 1M & $277.6687 \pm 52.1335$ & 0.0015 & 3 \\
3M & 1.0 & HF & CISD & 150 smallest CISD & 40k & 1M & $0.0306 \pm 0.0146$ & 19612 & 3 \\
5M & 1.0 & HF & CISD & 150 smallest CISD & 40k & 769k & $0.0141 \pm 0.0098$ & 57082 & 3 \\
\end{tabular}

\section*{FCI-FRI Calculations with the Alternative HB-PP Factorization}

The following results were obtained using the systematic FCI-FRI method with the alternative, ``un-normalized'' HB-PP factorization, without the semi-stochastic extension.

\subsection*{Ne (aug-cc-pVDZ)}
These results are for the Neon atom in the aug-cc-pVDZ basis, using the same Hartree-Fock data described above. As an example, the results in the first line of this table were generated by running the following command:

\texttt{mpirun -n 4 ./build/FRIES\_bin/frisys\_mol --hf\_path Input\_Data/Neon\_augccpvdz/ --target 50000 --distribution HB\_unnorm --vec\_nonz 50000 --mat\_nonz 50000 --max\_dets 50000 --initiator 1.0 --ini\_vec Input\_Data/Neon\_augccpvdz/cisd\_}
\\~\\
\begin{tabular}{c|c|c|c|c|c|c|c|c}
Nonz. & Ini. thresh. ($n_a$) & Trial vec. & Ini vec. & Burn-in & Iterations & Mean $\pm 2 \sigma$ (m$E_h$) & Efficiency ($E_h^{-2}$) & Figures \\ \hline
50k & 1.0 & HF & CISD & 40k & 1M & $0.0073 \pm 0.0079$ & 67108 & 4 \\
100k & 1.0 & HF & CISD & 40k & 1M & $0.0114 \pm 0.0056$ & 132196 & 4 \\
150k & 1.0 & HF & CISD & 40k & 1M & $0.0098 \pm 0.0042$ & 239832 & 4 \\
200k & 1.0 & HF & CISD & 40k & 1M & $0.0048 \pm 0.0035$ & 333066 & 4 \\
\end{tabular}


\subsection*{\ce{H2O} (cc-pVDZ)}
These results are for \ce{H2O} in the cc-pVDZ basis, using the same Hartree-Fock data described above. As an example, the results in the first line of this table were generated by running the following command:

\texttt{mpirun -n 16 ./build/FRIES\_bin/frisys\_mol --hf\_path Input\_Data/H2O\_ccpvdz/ --target 1000000 --distribution HB\_unnorm}\\ \texttt{--vec\_nonz 1000000 --mat\_nonz 1000000 --max\_dets 100000 --initiator 1.0 --ini\_vec Input\_Data/H2O\_ccpvdz/cisd\_}
\\~\\
\begin{tabular}{c|c|c|c|c|c|c|c|c}
Nonz. & Ini. thresh. ($n_a$) & Trial vec. & Ini vec. & Burn-in & Iterations & Mean $\pm 2 \sigma$ (m$E_h$) & Efficiency ($E_h^{-2}$) & Figures \\ \hline
1M & 1 & HF & CISD & 40k & 1M & $0.0283 \pm 0.0058$ & 122875 & 4\\
3M & 1 & HF & CISD & 40k & 1M & $0.0184 \pm 0.0027$ & 583227 & 4\\
5M & 1 & HF & CISD & 40k & 1M & $0.0167 \pm 0.0023$ & 1148398 & 4\\
\end{tabular}

\subsection*{\ce{N2} (equilibrium, cc-pVDZ)}
These results are for equilibrium \ce{N2} in the cc-pVDZ basis, using the same Hartree-Fock data described above. As an example, the results in the first line of this table were generated by running the following command:

\texttt{mpirun -n 20 ./build/FRIES\_bin/frisys\_mol --hf\_path Input\_Data/N2\_ccpvdz/ --target 1000000 --distribution HB\_unnorm}\\ \texttt{--vec\_nonz 1000000 --mat\_nonz 1000000 --max\_dets 100000 --initiator 1.0 --ini\_vec Input\_Data/N2\_ccpvdz/cisd\_}
\\~\\
\begin{tabular}{c|c|c|c|c|c|c|c|c}
Nonz. & Ini. thresh. ($n_a$) & Trial vec. & Ini vec. & Burn-in & Iterations & Mean $\pm 2 \sigma$ (m$E_h$) & Efficiency ($E_h^{-2}$) & Figures \\ \hline
1M & 1 & HF & CISD & 40k & 1M & $0.0697 \pm 0.0179$ & 13025 & 4\\
3M & 1 & HF & CISD & 40k & 1M & $0.0262 \pm 0.0080$ & 65667 & 4\\
5M & 1 & HF & CISD & 40k & 1M & $0.0154 \pm 0.0058$ & 146051 & 4\\
\end{tabular}

\subsection*{Ne (aug-pVQZ)}
These results are for Ne in the cc-pVQZ basis. The Hartree-Fock data for this system, generated using pyscf, can be found in the ZIP archive at \texttt{Input\_Data/Neon\_ccpvqz.zip}. After it is unzipped, the results in this table can be generated by running the following command:

\texttt{mpirun -n 10 ./build/FRIES\_bin/frisys\_mol --hf\_path Input\_Data/Neon\_ccpvqz/ --target 500000 --distribution HB\_unnorm --vec\_nonz 500000 --mat\_nonz 500000 --max\_dets 60000 --initiator 1}
\\~\\
\begin{tabular}{c|c|c|c|c|c|c|c|c}
Nonz. & Ini. thresh. ($n_a$) & Trial vec. & Ini vec. & Burn-in & Iterations & Mean $\pm 2 \sigma$ (m$E_h$) & Efficiency ($E_h^{-2}$) & Figures \\ \hline
500k & 1 & HF & HF & 40k & 1M & $-333.4149 \pm 0.0167$ & 14967 & - \\
\end{tabular}

\subsection*{\ce{N2} (stretched, cc-pVDZ)}
These results are for stretched \ce{N2} ($r_\text{NN} = 4.2 a_0$) in the cc-pVDZ basis. The Hartree-Fock data for this system, generated using pyscf, can be found in the directory \texttt{Input\_Data/N2\_str\_ccpvdz}. The results in this table were generated by running the following command:

\texttt{mpirun -n 20 ./build/FRIES\_bin/frisys\_mol --hf\_path Input\_Data/N2\_str\_ccpvdz/ --target 5000000 --distribution HB\_unnorm} \texttt{--vec\_nonz 5000000 --mat\_nonz 5000000 --max\_dets 500000 --initiator 3 --ini\_vec Input\_Data/N2\_str\_ccpvdz/cisd\_}
\\~\\
\begin{tabular}{c|c|c|c|c|c|c|c|c|c|c}
Vec. nonz. & Matr. nonz. & Target & $n_a$ & Trial vec. & Ini vec. & Burn-in & Iterations & Mean $\pm 2 \sigma$ (m$E_h$) & Efficiency ($E_h^{-2}$) & Figures \\ \hline
500k & 500k & 500k & 3 & HF & CISD & 80k & 350k & $0.1025 \pm 0.1019$ & 1205 & - \\
\end{tabular}

\section*{FCIQMC (integer) Calculations}
These results were obtained using our implementation of the FCIQMC method in which vector elements are constrained to be integers. The heat-bath Power-Pitzer matrix factorization was used in these calculations, along with the initiator approximation. The semi-stochastic extension was not used.

\subsection*{Ne (aug-cc-pVDZ)}
These results are for the Neon atom in the aug-cc-pVDZ basis, using the same Hartree-Fock data described above. All calculations were initialized from a population of 5000 walkers distributed among determinants in proportion to the values of elements in the CISD unit vector. As an example, the results in the first line of this table were generated by running the following command:

\texttt{mpirun -n 4 ./build/FRIES\_bin/fciqmc\_mol --hf\_path Input\_Data/Ne\_augccpvdz/ --target 50000 --initiator 3 --max\_dets 20000 --distribution HB --ini\_vec Input\_Data/Neon\_augccpvdz/cisd\_int\_}
\\~\\
\begin{tabular}{c|c|c|c|c|c|c|c|c}
Target walkers & Ini. thresh. ($n_a$) & Trial vec. & Ini vec. & Burn-in & Iterations & Mean $\pm 2 \sigma$ (m$E_h$) & Efficiency ($E_h^{-2}$) & Figures \\ \hline
50k & 3 & HF & CISD & 40k & 1M & $0.0099 \pm 0.0734$ & 773 & 5 \\
100k & 3 & HF & CISD & 40k & 1M & $-0.0136 \pm 0.0546$ & 1400 & 5 \\
150k & 3 & HF & CISD & 40k & 1M & $-0.0062 \pm 0.0457$ & 1998 & 5 \\
200k & 3 & HF & CISD & 40k & 1M & $0.0178 \pm 0.0382$ & 2851 & 5 \\
\end{tabular}


\subsection*{\ce{H2O} (cc-pVDZ)}
These results are for \ce{H2O} in the cc-pVDZ basis, using the same Hartree-Fock data described above. These calculations were initialized from a population of 500,000 walkers distributed among determinants in proportion to the values of elements in the CISD unit vector. As an example, the results in the first line of this table were generated by running the following command:

\texttt{mpirun -n 16 ./build/FRIES\_bin/fciqmc\_mol --hf\_path Input\_Data/H2O\_ccpvdz/ --target 1000000 --initiator 3 --max\_dets 100000 --distribution HB --ini\_vec Input\_Data/H2O\_ccpvdz/cisd\_int\_}
\\~\\
\begin{tabular}{c|c|c|c|c|c|c|c|c}
Target walkers & Ini. thresh. ($n_a$) & Trial vec. & Ini vec. & Burn-in & Iterations & Mean $\pm 2 \sigma$ (m$E_h$) & Efficiency ($E_h^{-2}$) & Figures \\ \hline
1M & 3 & HF & CISD & 40k & 1M & $0.0491 \pm 0.0332$ & 3770 & 5\\
10M & 3 & HF & CISD & 40k & 574k & $0.0120 \pm 0.0147$ & 34817 & 5\\
50M & 3 & HF & CISD & 40k & 339k & $0.0147 \pm 0.0022$ & 2403490 & 5 \\
\end{tabular}

\subsection*{\ce{N2} (equilibrium, cc-pVDZ)}
These results are for equilibrium \ce{N2} in the cc-pVDZ basis, using the same Hartree-Fock data described above. These calculations were initialized from a population of 500,000 walkers distributed among determinants in proportion to the values of elements in the CISD unit vector. As an example, the results in the first line of this table were generated by running the following command:

\texttt{mpirun -n 8 ./build/FRIES\_bin/fciqmc\_mol --hf\_path Input\_Data/N2\_ccpvdz/ --target 1000000 --initiator 3 --max\_dets 200000 --distribution HB --ini\_vec Input\_Data/N2\_ccpvdz/cisd\_int\_}
\\~\\
\begin{tabular}{c|c|c|c|c|c|c|c|c}
Target walkers & Ini. thresh. ($n_a$) & Trial vec. & Ini vec. & Burn-in & Iterations & Mean $\pm 2 \sigma$ (m$E_h$) & Efficiency ($E_h^{-2}$) & Figures \\ \hline
1M & 3 & HF & CISD & 40k & 1M & $0.1638 \pm 0.0781$ & 684 & 5\\
10M & 3 & HF & CISD & 40k & 543k & $0.0061 \pm 0.0398$ & 5010 & 5\\
50M & 3 & HF & CISD & 40k & 462k & $0.0089 \pm 0.0189$ & 24518 & 5 \\
\end{tabular}


\section*{FCIQMC (non-integer) Calculations}
These results were obtained using our implementation of the FCIQMC method in which only some vector elements are integerized, and the rest are allowed to be non-integers. The heat-bath Power-Pitzer matrix factorization was used in these calculations, along with the initiator approximation. The semi-stochastic extension was not used.

\subsection*{Ne (aug-cc-pVDZ)}
These results are for the Neon atom in the aug-cc-pVDZ basis, using the same Hartree-Fock data and initial vector as described in the previous FCIQMC section. As an example, the results in the first line of this table were generated by running the following command:

\texttt{mpirun -n 4 ./build/FRIES\_bin/fciqmc\_fp\_mol --hf\_path Input\_Data/Ne\_augccpvdz/ --target 50000 --initiator 3 --max\_dets 20000 --distribution HB --ini\_vec Input\_Data/Neon\_augccpvdz/cisd\_int\_}
\\~\\
\begin{tabular}{c|c|c|c|c|c|c|c|c}
Target walkers & Ini. thresh. ($n_a$) & Trial vec. & Ini vec. & Burn-in & Iterations & Mean $\pm 2 \sigma$ (m$E_h$) & Efficiency ($E_h^{-2}$) & Figures \\ \hline
50k & 3 & HF & CISD & 40k & 1M & $0.0307 \pm 0.0176$ & 13390 & 5 \\
100k & 3 & HF & CISD & 40k & 1M & $0.0313 \pm 0.0126$ & 26240 & 5 \\
150k & 3 & HF & CISD & 40k & 1M & $0.0336 \pm 0.0103$ & 39821 & 5 \\
200k & 3 & HF & CISD & 40k & 1M & $0.0290 \pm 0.0090$ & 51644 & 5 \\
\end{tabular}


\subsection*{\ce{H2O} (cc-pVDZ)}
These results are for \ce{H2O} in the cc-pVDZ basis, using the same Hartree-Fock data and initial vector as described in the previous FCIQMC section. As an example, the results in the first line of this table were generated by running the following command:

\texttt{mpirun -n 16 ./build/FRIES\_bin/fciqmc\_fp\_mol --hf\_path Input\_Data/H2O\_ccpvdz/ --target 1000000 --initiator 3 --max\_dets 100000 --distribution HB --ini\_vec Input\_Data/H2O\_ccpvdz/cisd\_int\_}
\\~\\
\begin{tabular}{c|c|c|c|c|c|c|c|c}
Target walkers & Ini. thresh. ($n_a$) & Trial vec. & Ini vec. & Burn-in & Iterations & Mean $\pm 2 \sigma$ (m$E_h$) & Efficiency ($E_h^{-2}$) & Figures \\ \hline
1M & 3 & HF & CISD & 40k & 1M & $0.0404 \pm 0.0097$ & 44495 & 5\\
10M & 3 & HF & CISD & 40k & 478k & $0.0165 \pm 0.0045$ & 442339 & 5\\
50M & 3 & HF & CISD & 5k & 334k & $0.0157 \pm 0.0021$ & 2818118 & 5 \\
\end{tabular}

\subsection*{\ce{N2} (equilibrium, cc-pVDZ)}
These results are for equilibrium \ce{N2} in the cc-pVDZ basis, using the same Hartree-Fock data and initial vector as described in the previous FCIQMC section. As an example, the results in the first line of this table were generated by running the following command:

\texttt{mpirun -n 8 ./build/FRIES\_bin/fciqmc\_fp\_mol --hf\_path Input\_Data/N2\_ccpvdz/ --target 1000000 --initiator 3 --max\_dets 200000 --distribution HB --ini\_vec Input\_Data/N2\_ccpvdz/cisd\_int\_}
\\~\\
\begin{tabular}{c|c|c|c|c|c|c|c|c}
Target walkers & Ini. thresh. ($n_a$) & Trial vec. & Ini vec. & Burn-in & Iterations & Mean $\pm 2 \sigma$ (m$E_h$) & Efficiency ($E_h^{-2}$) & Figures \\ \hline
1M & 3 & HF & CISD & 40k & 1M & $0.1316 \pm 0.0190$ & 10574 & 5\\
10M & 3 & HF & CISD & 40k & 438k & $0.0095 \pm 0.0099$ & 103387 & 5\\
50M & 3 & HF & CISD & 40k & 307k & $0.0097 \pm 0.0060$ & 422516 & 5 \\
\end{tabular}


\end{document}